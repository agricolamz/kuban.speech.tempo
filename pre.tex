%!TEX TS-program = xelatex

\documentclass[a4paper,12pt, landscape]{article}

\usepackage[english,russian]{babel}   %% загружает пакет многоязыковой вёрстки
\usepackage{fontspec}      %% подготавливает загрузку шрифтов Open Type, True Type и др.
\defaultfontfeatures{Ligatures={TeX},Renderer=Basic}  %% свойства шрифтов по умолчанию
\setmainfont[Ligatures={TeX,Historic},
SmallCapsFont={Brill},
SmallCapsFeatures={Letters=SmallCaps}]{Brill} %% задаёт основной шрифт документа
\setsansfont{Brill}                    %% задаёт шрифт без засечек
\setmonofont{Noto Sans Ethiopic}
\usepackage{indentfirst}

%%% Дополнительная работа с математикой
\usepackage{amsmath,amsfonts,amssymb,amsthm,mathtools} % AMS
\usepackage{icomma} % "Умная" запятая: $0,2$ --- число, $0, 2$ --- перечисление

%%% Работа с картинками
\usepackage{wrapfig} % Обтекание рисунков текстом
\usepackage{subcaption}
\usepackage{rotating}
\usepackage{fixltx2e}
\usepackage{hhline}
\usepackage{lscape}
\usepackage[usenames,dvipsnames,svgnames,table,rgb]{xcolor}%пакет для использования цветов 

\usepackage{enumitem}
\setlist{nolistsep, leftmargin=5mm}

%%% Работа с таблицами
\usepackage{array,tabularx,tabulary,booktabs} % Дополнительная работа с таблицами
\usepackage{longtable}  % Длинные таблицы
\usepackage{multirow} % Слияние строк в таблице

\usepackage{multicol} % Несколько колонок

%%% Страница
\usepackage{extsizes} % Возможность сделать 14-й шрифт
\usepackage[twocolumn]{geometry} % Простой способ задавать поля
	\geometry{top=12mm}
	\geometry{bottom=13mm}
	\geometry{left=13mm}
	\geometry{right=9mm}

%\usepackage{fancyhdr} % Колонтитулы
% 	\pagestyle{fancy}
 	%\renewcommand{\headrulewidth}{0pt}  % Толщина линейки, отчеркивающей верхний колонтитул
% 	\lfoot{Нижний левый}
% 	\rfoot{Нижний правый}
% 	\rhead{Верхний правый}
% 	\chead{Верхний в центре}
% 	\lhead{Верхний левый}
%	\cfoot{Нижний в центре} % По умолчанию здесь номер страницы

\usepackage{setspace} % Интерлиньяж
%\onehalfspacing % Интерлиньяж 1.5
%\doublespacing % Интерлиньяж 2
\singlespacing % Интерлиньяж 1

\usepackage{lastpage} % Узнать, сколько всего страниц в документе.
\usepackage{soul} % Модификаторы начертания
\usepackage{bbding}
\usepackage{hyperref}
\usepackage[usenames,dvipsnames,svgnames,table,rgb]{xcolor}
\hypersetup{				% Гиперссылки
    colorlinks=true,       	% false: ссылки в рамках; true: цветные ссылки
    linkcolor=black,          % внутренние ссылки
    citecolor=black,        % на библиографию
    filecolor=black,      % на файлы
    urlcolor=ForestGreen          % на URL
}

\usepackage{environ}
\makeatletter
\newsavebox{\measure@tikzpicture}
\NewEnviron{scaletikzpicturetowidth}[1]{%
	\def\tikz@width{#1}%
	\def\tikzscale{1}\begin{lrbox}{\measure@tikzpicture}%
		\BODY
	\end{lrbox}%
	\pgfmathparse{#1/\wd\measure@tikzpicture}%
	\edef\tikzscale{\pgfmathresult}%
	\BODY
}
\makeatother

\usepackage{pgfplots}
\usepackage{pgfplotstable}
\usepackage{verbatim}

\usepackage{attachfile2}
 \attachfilesetup{appearance=true,
color=0 0 0
 }

%%% Лингвистические пакеты
%\usepackage{savetrees} % пакет, который экономит место
\usepackage{forest} % для рисования деревьев
\usepackage{vowel} % для рисования трапеций гласных
\usepackage{natbib}
\bibpunct[: ]{[}{]}{;}{a}{}{,}
\usepackage[nogroupskip,nopostdot, nonumberlist]{glossaries}
%\usepackage{glossary-mcols} 
%\setglossarystyle{mcolindex}
\usepackage{philex} % пакет для примеров
\addto\captionsrussian{% Replace "english" with the language you use
\renewcommand{\refname}{}
\renewcommand{\glossaryname}{Список глосс}
}
\newcommand{\mytem}{\item[$\circ$]}

\usepackage{todonotes}
\newcounter{mycomment}
\newcommand{\mycom}[1]{
\refstepcounter{mycomment}%
{%
\setstretch{0.7}% spacing
\todo[color=blue!20!white, inline]{%
\textbf{ГМ\themycomment:}~{\footnotesize #1}}%
}}
\renewcommand{\thesection}{\arabic{section}.}
\renewcommand{\thesubsection}{\arabic{section}.\arabic{subsection}}
\setlength{\columnsep}{1.6cm}

\usepackage{sectsty}
\sectionfont{\normalsize}
\subsectionfont{\normalsize}
\usepackage{titlesec}
\titlespacing*{\section}
{0pt}{2ex plus 0ex minus .2ex}{0ex plus .2ex}
\titlespacing*{\subsection}
{0pt}{2ex plus 0ex minus .2ex}{0ex plus .2ex}
\newlength{\bibitemsep}\setlength{\bibitemsep}{.2\baselineskip plus .05\baselineskip minus .05\baselineskip}
\newlength{\bibparskip}\setlength{\bibparskip}{0pt}
\let\oldthebibliography\thebibliography
\renewcommand\thebibliography[1]{%
	\oldthebibliography{#1}%
	\setlength{\parskip}{\bibitemsep}%
	\setlength{\itemsep}{\bibparskip}%
}