\noindent При чтении прозаического фрагмента использовалась история составленная с информантами: \medskip\\
\noindent Зы махуэ гуэрэм сэ Лабинск сыкIуэт. Хуэбэ Iет. Маршруткэм сису, жъышIэхуыр сIыгъу жъы сеуэт. Телефоныр къытеуа. Телефоныр къытесха ар Аслъэн къэзышIыр.\\
– Дыгъуасэ, уэ си машинэ IункIыбзэр Аминэрэ Бислъанрэ ириджэгуну яптатэкъэ? – жьери къызэуыпшIа.\\
– Ае, – жьесIэжьа.\\
– Тэнэ ар здахьар?\\
– Уэуи, сабихэр пщахъуэм шъыджэгуахэт, абым къыханауэ шътын. Уынэм къышъышIэхьэжьхэм уиIункIыбзэр яIыгъу слъэгъуакъым.\\
– Тэнэ джы здэкIуар тIэ сиIункIыбзэр? – Аслъан  къышIэуыпшIа.\\
– Тэнэ сэ шъысцIыхуыр? Нэнаухэм пщахъуэм Аслъан иIункIыбзэр шIэтIтIат жьаIу къызжьаIа пфIэшIрэ?\\
– ПсынчIу, псынчIу къэгъуэтыжь, сэ Майкоп сыкIуэн хует нэIэ джыпсту, шъхьакIэ сиIункIыбзэр вгъэкIуэда.\\
– Ой сэ Лабинск сокIуэ джыпсту, къысхуэгъэгъу. Сиде Iуыхьи, уынэм шIыхьи сабихэм яупшI. КъагъуэтыжьынкIи мэхъу.\bigskip\\

\textbf{Перевод:}\\
Однажды я ехала в Лабинск. Жара стращная. Я сижу в маршрутке, в руках веер, обмахиваюсь. Тут телефон звонит. Оказалось это Аслан.\\
--- Ты давала вчера мои ключи от машины Амине и Бислану поиграть? --- спросил он.\\
--- Да, --- отвечаю.\\
--- Ну и куда они их дели?\\
--- Ой, дети играли в песочнице, там, наверное, и оставили. Я не видела ключей, когда они в дом зашли.\\
--- Ну и где тогда теперь мои ключи?\\
--- Откуда я знаю? Думаешь, они пришли и сказали "мы закопали ключи Аслана".
--- Быстрее, быстрее, найди их, я же в Майкоп поехать не могу, из-за того, что вы ключи потеряли.\\
--- Ой, я в Лабинск сейчас еду, прости. Ко мне домой зайди, детей спроси. Может найдутся.
\pagebreak